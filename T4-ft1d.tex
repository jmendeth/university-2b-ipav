\input{preamble.tex}
\AfterPreamble{\hypersetup{
  pdftitle={Entregable 4: Análisis frecuencial 1D},
}}

\begin{document}
\startpage

\paragraph{Problema 1.}

Aquí lo haremos mediante la definición.

\subparagraph{Apartado A.}

\begin{align*}
  \mathcal{F}\{ x[n_0 - n] \}(F)
  &= \sum_{n = -\infty}^{\infty} x[n_0 - n] e^{-j\tau F n}
\end{align*}
%
Aplicamos un cambio de variable $m = n_0 - n$; el intervalo del sumatorio no
cambia:
%
\begin{align*}
  \mathcal{F}\{ x[n_0 - n] \}(F)
  &= \sum_{m = -\infty}^{\infty} x[m] e^{-j\tau F \left( n_0 - m \right) }
\\
  \mathcal{F}\{ x[n_0 - n] \}(F)
  &= \sum_{m = -\infty}^{\infty} x[m] e^{+j\tau F m }
    \, e^{-j\tau F n_0 }
\\
  \mathcal{F}\{ x[n_0 - n] \}(F)
  &= \mathcal{F}\{ x[n] \}(-F) \, e^{-j\tau F n_0 } = X(-F)\, e^{-j\tau F n_0 }
\end{align*}

\subparagraph{Apartado B.}

\begin{align*}
  \mathcal{F}\{ x^*[n + k] \}(F)
  &= \sum_{n = -\infty}^{\infty} x^*[n + k] e^{-j\tau F n}
\end{align*}
%
Aplicamos un cambio de variable $m = n + k$; el intervalo del sumatorio no
cambia:
%
\begin{align*}
  \mathcal{F}\{ x^*[n + k] \}(F)
  &= \sum_{m = -\infty}^{\infty} x^*[m] e^{-j\tau F \left( m - k \right) }
\\
  \mathcal{F}\{ x^*[n + k] \}(F)
  &= \sum_{m = -\infty}^{\infty} x^*[m] e^{-j\tau F m }
    \, e^{j\tau F k }
\\
  \mathcal{F}\{ x^*[n + k] \}(F)
  &= \left[ \sum_{m = -\infty}^{\infty} x[m] e^{+j\tau F m } \right]^*
    \, e^{j\tau F k }
\\
  \mathcal{F}\{ x^*[n + k] \}(F)
  &= \left[ \mathcal{F}\{ x[n] \}(-F) \right]^*
    \, e^{j\tau F k } = X^*(-F)\, e^{j\tau F k }
\end{align*}

\finishpage
\startpage

\paragraph{Problema 2.}

\subparagraph{Apartado A.}

El primer periodo de $x[n]$ es $\{ 1, 0, -1, 0 \}$, por tanto usando la
definición queda:
%
\begin{align*}
  F_p(F) &= 1\cdot e^{-j\tau F \cdot 0} -1\cdot e^{-j\tau F \cdot 2}
    = 1 - e^{-j\tau 2 F}
\end{align*}

\subparagraph{Apartado B.}

Dado que tanto la DFT como el periodo son de 4 muestras, la DFT es simplemente
la transformada anterior muestreada adecuadamente:
%
\begin{align*}
  X[k] &= F_p( \nicefrac{k}{4} ) = 1 - e^{-j\tau \nicefrac{k}{2}} =
    \{ 0, 2, 0, 2 \}
\end{align*}

\subparagraph{Apartado C.}

La transformada de Fourier de la señal periódica es simplemente la DFT
de un periodo (apartado anterior) expresada como distribución:
%
\begin{align*}
  X(F) = \sum_{k = -\infty}^{\infty} X[k] \,
    \delta\hspace{-.2em}\left(4F - k\right)
    = \nicefrac{1}{2} \, \Psi_{\nicefrac{1}{2}}(F - \nicefrac{1}{4})
\end{align*}

\subparagraph{Apartado D.}

La antitransformada de la DFT nos dará la señal original expresada como suma
de exponenciales tal y como se pide:
%
\begin{align*}
  x[n] &= \frac{1}{4} \sum_{k = 0}^{3} X[k] e^{j\tau \frac{k}{4} n}
    = \frac{1}{2} e^{j\tau \frac{n}{4}}
    + \frac{1}{2} e^{-j\tau \frac{n}{4}}
\end{align*}

\finishpage
\startpage

\paragraph{Problema 3.}

\subparagraph{Apartado A.}

La ventana es rectangular, con lo cual la anchura de los lóbulos (mínima
diferencia en frecuencia que podemos discernir) es $\Delta F = \nicefrac{2}{L}$.
Dado que queremos garantizar $\Delta F < \frac{\SI{1.9}{\hertz}}
{\SI{130}{\hertz}}$, se deduce que $L >
2\frac{\SI{130}{\hertz}}{\SI{1.9}{\hertz}} \Rightarrow L \geq \num{137}$.

La sensibilidad no nos limita en este caso porque por la naturaleza de un
acorde las tres notas son de similar amplitud.

\subparagraph{Apartado B.}

De entrada, observamos unicamente tres 'deltas' en la primera FFT. Con lo cual
podemos garantizar que durante esta ventana solo se ha tocado un acorde con
esas notas.

Nos fijamos en los bins de las tres notas y al pasarlos a frecuencia analógica
obtenemos aproximadamente \SI{32.7}{\hertz} (\num{515}), \SI{41.2}{\hertz}
(\num{649}) y \SI{49}{\hertz} (\num{772}). Esto corresponde a las notas
Do\textsubscript{1}, Mi\textsubscript{1} y Sol\textsubscript{1}; sería un
acorde Do mayor.

En la segunda FFT vemos contribuciones en exactamente las mismas notas que
antes, junto con otras dos nuevas. Si nos fijamos en las amplitudes lo podemos
interpretar como la suma de la triada anterior, y otra nueva triada que
comparten una de las notas.

Las nuevas notas, siguiendo el procedimiento anterior, tienen frecuencias
aproximadas de \SI{41.2}{\hertz} (\num{649}), \SI{51.9}{\hertz} (\num{818}) y
\SI{61.8}{\hertz} (\num{973}). Esto corresponde a las notas
Mi\textsubscript{1}, Sol\#\textsubscript{1} y Si\textsubscript{1}; sería un
acorde Mi mayor.

Las anchuras de los lóbulos de ambas triadas de la segunda ventana son las
mismas por lo que podemos estar razonablemente seguros de las dos ocupan la
mitad de la ventana en tiempo. Es importante tener en cuenta que no podemos
saber en qué orden se tocan (esa información vendría dada por la fase).

Una interpretación posible es que la duración de las triadas es de media
ventana, y que en la primera ventana había dos triadas iguales que se analizan
como una. En ese caso cada triada dura $D = \num{0.5} \frac{256}{130}
\approx \SI{0.98}{\second}$.

\finishpage
\startpage

\paragraph{Problema 4.}

\subparagraph{Apartado A.}

Para evitar aliasing se debe usar una frecuencia de muestreo superior al doble
de la máxima frecuencia de trabajo, por tanto en este caso $f_s >
\SI{50}{\hertz}$.

\subparagraph{Apartado B.}

Determinamos los bins de las tres 'deltas' en la DFT y sus correspondientes
frecuencias discretas aproximadas: \num{0.1} (\num{41}), \num{0.2} (\num{81})
y \num{0.3} (\num{120}).

Las frecuencias analógicas según una frecuencia de muestreo de \SI{50}{\hertz}
son, respectivamente: \SI{5}{\hertz}, \SI{10}{\hertz} y \SI{15}{\hertz}.

\subparagraph{Apartado C.}

La duración se relaciona con la anchura del lóbulo según la expresión:
$L_1 = \frac{2N}{k}$

Respecto al primer tono, permanece hasta la segunda ventana donde tiene una
anchura de unas 8 muestras. Esto significa $L_1 \simeq L + \frac{2\cdot 400}{8}
\simeq 200$ muestras (dura toda la señal).

Respecto al segundo tono, permanece hasta la segunda ventana donde tiene una
anchura de unas 33 muestras. Esto significa $L_1 \simeq L + \frac{2\cdot 400}{33}
\simeq 125$ muestras.

Respecto al tercer tono, este muere ya en la primera ventana donde tiene una
anchura de unas 10 muestras. Esto significa $L_1 \simeq \frac{2\cdot 400}{10}
\simeq 80$ muestras.


\finishpage
\end{document}
