\documentclass[catalan,border=15pt,class=scrartcl,multi=minipage,parskip=half*]{standalone}

% encoding
\usepackage[utf8]{inputenc}
\usepackage[T1]{fontenc}
\usepackage{lmodern}
\usepackage{babel}

% formatting and fixes
\frenchspacing
\usepackage[style=spanish]{csquotes}
\MakeAutoQuote{«}{»}
\usepackage{bookmark}

% ADD ANY SPECIFIC PACKAGES HERE
% (CHEMISTRY, CODE, PUBLISHING)
\usepackage[usenames,dvipsnames,svgnames,table]{xcolor}
\usepackage{adjustbox}
\usepackage{booktabs}
\usepackage{mathtools}
\usepackage{commath}
\usepackage{tikz}
\usepackage{siunitx}
\usepackage{nicefrac}
\usetikzlibrary{calc}
\usetikzlibrary{arrows.meta}
\usetikzlibrary{automata}
\usepackage{minted}

% other options
\setcounter{tocdepth}{6}
\setcounter{secnumdepth}{2}

% hyperlink setup / metadata
\usepackage{hyperref}
\AfterPreamble{\hypersetup{
  pdfauthor={Xavier Mendez},
  pdfsubject={IPAV},
  pdfpagelayout=OneColumn,
}}

% custom commands
\newcommand{\startpage}{\begin{minipage}{30em} \setlength{\parskip}{0.5em}}
\newcommand{\finishpage}{\end{minipage}}
\newcommand{\iopair}[2]{\( \left(#1\right) \rightarrow #2 \)}

\AfterPreamble{\hypersetup{
  pdftitle={Entregable 5: Análisis frecuencial 2D},
}}

\begin{document}
\startpage

\paragraph{Problema 1.}

Asumiremos que las imagenes tienen el origen de coordenadadas situado en el centro.

\subparagraph{Apartado A.}

Podemos aplicar la definición sobre la imagen B expresada en términos de la imagen A:
%
\begin{align*}
  \mathcal{F}\left\{ x_B[m, n] \right\}(F_1, F_2) &=
  \sum_{m=-\infty}^{\infty} e^{-j\tau F_1 m} \,
  \sum_{n=-\infty}^{\infty} e^{-j\tau F_2 n} \,
  x_B[m, n]
\\
  X_B(F_1, F_2) &=
  \sum_{m=-\infty}^{\infty} e^{-j\tau F_1 m} \,
  \sum_{n=-\infty}^{\infty} e^{-j\tau F_2 n} \,
  x_A[-m, n]
\end{align*}
%
Aplicamos un cambio de variable en el sumatorio externo, manipulamos y reconocemos
la transformada de $x_A$:
%
\begin{align*}
  X_B(F_1, F_2) &=
  \sum_{m'=-\infty}^{\infty} e^{-j\tau F_1 (-m')} \,
  \sum_{n=-\infty}^{\infty} e^{-j\tau F_2 n} \,
  x_A[m', n]
\\
  X_B(F_1, F_2) &=
  \sum_{m'=-\infty}^{\infty} e^{-j\tau (-F_1) m'} \,
  \sum_{n=-\infty}^{\infty} e^{-j\tau F_2 n} \,
  x_A[m', n]
\\
  X_B(F_1, F_2) &= X_A(-F_1, F_2)
\end{align*}

\subparagraph{Apartado B.}

Para la imagen C seguimos un procedimiento análogo:
%
\begin{align*}
  \mathcal{F}\left\{ x_C[m, n] \right\}(F_1, F_2) &=
  \sum_{m=-\infty}^{\infty} e^{-j\tau F_1 m} \,
  \sum_{n=-\infty}^{\infty} e^{-j\tau F_2 n} \,
  x_A[-m, -n]
\\
  X_C(F_1, F_2) &=
  \sum_{m'=-\infty}^{\infty} e^{-j\tau F_1 (-m')} \,
  \sum_{n=-\infty}^{\infty} e^{-j\tau F_2 n} \,
  x_A[m', -n]
\\
  X_C(F_1, F_2) &=
  \sum_{m'=-\infty}^{\infty} e^{-j\tau F_1 (-m')} \,
  \sum_{n'=-\infty}^{\infty} e^{-j\tau F_2 (-n')} \,
  x_A[m', n']
\\
  X_C(F_1, F_2) &= X_A(-F_1, -F_2)
\end{align*}

\subparagraph{Apartado C.}

De nuevo usamos la definición de la antitransformada:
%
\begin{align*}
  \mathcal{F}^{-1}\left\{ X_A^*(F_1, F_2) \right\}(m, n) &=
  \iint_{-\infty}^{\infty} X_A^*(F_1, F_2) e^{j\tau\left[F_1 m + F_2 n\right]} \dif F_1 \dif F_2
\end{align*}
%
Y simplificamos hasta reconocer la antitransformada de $X_A(F_1, F_2)$:
%
\begin{align*}
  \quad &=
  \iint_{-\infty}^{\infty} X_A^*(F_1, F_2) e^{-j\tau\left[F_1 (-m) + F_2 (-n)\right]} \dif F_1 \dif F_2
\\
  \quad &=
  \iint_{-\infty}^{\infty} X_A^*(F_1, F_2) \left(e^{j\tau\left[F_1 (-m) + F_2 (-n)\right]}\right)^* \dif F_1 \dif F_2
\\
  \quad &=
  \left( \iint_{-\infty}^{\infty} X_A(F_1, F_2) e^{j\tau\left[F_1 (-m) + F_2 (-n)\right]} \dif F_1 \dif F_2 \right)^*
\\
  \quad &=
  \left( x_A(-m, -n) \right)^*
\end{align*}

Sabiendo que ambas antitransformadas son secuencias de igual periodo y
que son reales, la nueva secuencia es simplemente $x_A[-m, -n]$.

\finishpage


\startpage

\paragraph{Problema 2.}

\subparagraph{Apartado A.}

Aplicando mentalmente la definición y simplificando:
%
\begin{align*}
  X(F_1, F_2) &= e^{-j\tau\left[ 1F_1 + 0F_2 \right]} + e^{-j\tau\left[ 1F_1 + 1F_2 \right]}
\\
  X(F_1, F_2) &= e^{-j\tau\left[ 1F_1 + \frac{1}{2}F_2 \right]}
    \left( e^{+j\tau\frac{1}{2}F_2} + e^{-j\tau\frac{1}{2}F_2} \right)
\\
  X(F_1, F_2) &= 2 \cos(\tau\nicefrac{F_2}{2}) e^{-j\tau\left( F_1 + \frac{F_2}{2} \right)}
\end{align*}

\subparagraph{Apartado B.}

Una forma de hacerlo es aplicando la DFT por columnas:
%
\begin{equation*}
  \begin{bmatrix}
    0 & 2 \\
    0 & 0
  \end{bmatrix}
\end{equation*}
%
Y luego por filas:
%
\begin{equation*}
  \begin{bmatrix}
    2 & -2 \\
    0 & 0
  \end{bmatrix}
\end{equation*}
%
Esto coincide, si lo comprobamos, con la expresión obtenida en el apartado anterior muestreada
adecuadamente.

\subparagraph{Apartado C.}

La DFT es invariable delante de la transposición, así que el resultado es la misma matriz,
transpuesta:
%
\begin{equation*}
  \begin{bmatrix}
    2 & 0 \\
    -2 & 0
  \end{bmatrix}
\end{equation*}

\subparagraph{Apartado D.}

La DFT es lineal, así que el resultado es la misma matriz, multiplicada por 10:
%
\begin{equation*}
  \begin{bmatrix}
    20 & -20 \\
    0 & 0
  \end{bmatrix}
\end{equation*}

\finishpage


\startpage

\paragraph{Problema 3.}

\subparagraph{Apartado A.}

El proceso pone a 0 el bin $(0, 0)$ de la DFT, con lo cual establece la media de la imagen
a cero sin cambiar nada más. Es decir, sustrae una constante $k$ a todos los píxeles de la imagen
de forma que la suma de todos los nuevos píxeles sea nula:
%
\begin{align*}
  x'[m, n] = x[m, n] - k \\
  \sum_{m=0}^{M} \sum_{n=0}^{N} x'[m, n] = 0 \\
  \sum_{m=0}^{M} \sum_{n=0}^{N} x[m, n] - k = 0 \\
  \sum_{m=0}^{M} \sum_{n=0}^{N} x[m, n] = k \cdot MN \\
  k = \frac{1}{MN} \sum_{m=0}^{M} \sum_{n=0}^{N} x[m, n]
\end{align*}
%
En otras palabras: el proceso sustrae de cada píxel de la imagen, la media de dicha imagen.

\subparagraph{Apartado B.}

Dado que el proceso solo suma una constante a todos los píxeles, el margen dinámico no cambia
a menos que volvamos a cuantificar.

\subparagraph{Apartado C.}

Se obtiene $x[m, n] - x'[m, n] = x[m, n] - (x[m, n] - k) = k$ queda una imágen que tiene todos
los píxeles establecidos a $k$, la media de $x[m, n]$.

\finishpage


\startpage

\paragraph{Problema 4.}

\subparagraph{Apartado A.}

La matriz es $1$ salvo cuando $m + n - (N - 1) = 0$, en cuyo caso es $0$.
Para $M = N$, la matriz sería:
%
\begin{equation*}
  \begin{bmatrix}
    0 & 0 & 0 & 0 & \cdots & 0 & 0 & 1 \\
    0 & 0 & 0 & 0 &        & 0 & 1 & 0 \\
    0 & 0 & 0 & 0 &        & 1 & 0 & 0 \\
    0 & 0 & 0 & 0 &        & 0 & 0 & 0 \\

    \vdots  & & & & \ddots & & & \vdots\\

    0 & 0 & 1 & 0 &        & 0 & 0 & 0 \\
    0 & 1 & 0 & 0 &        & 0 & 0 & 0 \\
    1 & 0 & 0 & 0 & \cdots & 0 & 0 & 0
  \end{bmatrix}
\end{equation*}
%
Para obtener la matriz para $M < N$ o $M > N$, sustraer columnas del final o añadir
columnas nulas según convenga.

\subparagraph{Apartado B.}

Tenemos la matriz:
%
\begin{equation*}
  \begin{bmatrix}
    0 & 0 & 0 & 1 \\
    0 & 0 & 1 & 0 \\
    0 & 1 & 0 & 0 \\
    1 & 0 & 0 & 0
  \end{bmatrix}
\end{equation*}
%
Aplicamos DFT por filas:
%
\begin{equation*}
  \begin{bmatrix}
     1 &  j & -1 & -j \\
     1 & -1 &  1 & -1 \\
     1 & -j & -1 &  j \\
     1 &  1 &  1 &  1
  \end{bmatrix}
\end{equation*}
%
Ahora aplicamos DFT por columnas:
%
\begin{equation*}
  \begin{bmatrix}
     4 &  0 &  0 &  0 \\
     0 & 4j &  0 &  0 \\
     0 &  0 & -4 &  0 \\
     0 &  0 &  0 & -4j
  \end{bmatrix}
\end{equation*}
%

\finishpage
\end{document}
