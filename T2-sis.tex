\documentclass[catalan,border=15pt,class=scrartcl,multi=minipage,parskip=half*]{standalone}

% encoding
\usepackage[utf8]{inputenc}
\usepackage[T1]{fontenc}
\usepackage{lmodern}
\usepackage{babel}

% formatting and fixes
\frenchspacing
\usepackage[style=spanish]{csquotes}
\MakeAutoQuote{«}{»}
\usepackage{bookmark}

% ADD ANY SPECIFIC PACKAGES HERE
% (CHEMISTRY, CODE, PUBLISHING)
\usepackage[usenames,dvipsnames,svgnames,table]{xcolor}
\usepackage{adjustbox}
\usepackage{booktabs}
\usepackage{mathtools}
\usepackage{commath}
\usepackage{tikz}
\usepackage{siunitx}
\usepackage{nicefrac}
\usetikzlibrary{calc}
\usetikzlibrary{arrows.meta}
\usetikzlibrary{automata}
\usepackage{minted}

% other options
\setcounter{tocdepth}{6}
\setcounter{secnumdepth}{2}

% hyperlink setup / metadata
\usepackage{hyperref}
\AfterPreamble{\hypersetup{
  pdfauthor={Xavier Mendez},
  pdfsubject={IPAV},
  pdfpagelayout=OneColumn,
}}

% custom commands
\newcommand{\startpage}{\begin{minipage}{30em} \setlength{\parskip}{0.5em}}
\newcommand{\finishpage}{\end{minipage}}
\newcommand{\iopair}[2]{\( \left(#1\right) \rightarrow #2 \)}

\AfterPreamble{\hypersetup{
  pdftitle={Entregable 2: SiS, percepción y cuantificación},
}}

\begin{document}
\startpage

\paragraph{Problema 1.}

\begin{equation*}
  x[n] = \frac{1}{P} \sum_{m=0}^{P-1} e^{j\tau\frac{m}{P}n}
\end{equation*}

\subparagraph{Apartado A.}

Manipulamos para tener una serie geométrica y calculamos la suma:
%
\begin{align*}
  x[n] &= \frac{1}{P} \sum_{m=0}^{P-1} \left( e^{j\tau\frac{1}{P}n} \right)^{m}
\\
  x[n] &= \frac{1}{P} \, \frac{ e^{j\tau\frac{P}{P}n} - e^{j\tau\frac{0}{P}n} }{
    e^{j\tau\frac{1}{P}n} - 1 }
    \quad \text{si $e^{j\tau\frac{1}{P}n} \neq 1$}
\end{align*}
%
Simplificamos la expresión:
%
\begin{align*}
  x[n] &= \frac{1}{P} \, \frac{ 1 - 1 }{
    e^{j\tau\frac{1}{P}n} - 1 }
    \quad \text{excepto si $\tau \,|\, \tau\frac{1}{P}n$}
\\
  x[n] &= 0
    \quad \text{excepto si $P \,|\, n$}
\end{align*}

\subparagraph{Apartado B.}

No puedo particularizar la expresión resultante para $n$ siendo un múltiplo de
$P$ ya que la condición lo impide (tendríamos que aplicar la expresión
original: $ x[n] = \frac{1}{P} \, P = 1 $).

Respecto a $n \neq iP$, aquí sí aplica la expresión encontrada y tenemos $x[n]
= 0$. Teniendo esto en cuenta, $x[n]$ corresponde a un tren de pulsos ya que
todas las muestras son nulas excepto en $n$ múltiplo de $P$.

\finishpage


\startpage
\paragraph{Problema 2.}

\subparagraph{Apartado A.}

Se distinguen tres «deltas» diferenciadas en la DFT, cada una con su
equivalente simétrica al otro lado. Por tanto habría tres sinusoides en la
señal de entrada.

\subparagraph{Apartado B.}

Respecto a las frecuencias, solo tenemos que determinar los bins donde estan
situadas las deltas, y la frecuencia discreta correspondiente será
$F = \frac{n}{N}$. Los bins son aproximadamente \num{52}, \num{125} y \num{205}.
Ello daría frecuencias aproximadas de $F_1 = \nicefrac{1}{10}$, $F_2 = \nicefrac{1}{4}$ y $F_3 = \nicefrac{2}{5}$.

Respecto a las amplitudes, la amplitud de las dos sincs que una sinusoide
enventanada de amplitud 1 produce corresponde a $\nicefrac{L}{2}$ (media
longitud de la ventana). Las sincs correspondientes tienen amplitudes
aproximadas de \num{480}, \num{245} y \num{1200}. Eso daría amplitudes
aproximadas de $\nicefrac{15}{8}$, $\nicefrac{24}{25}$ y $\nicefrac{47}{10}$.

\finishpage


\startpage
\paragraph{Problema 3.}

La banda crítica de una cierta frecuencia $f$ es el rango de frecuencias que nuestro sistema auditivo no nos permite diferenciar de $f$. La curva de la
figura 2 indica que la banda crítica es aproximadamente \SI{100}{\hertz} para
frecuencias de hasta \SI{500}{\hertz}, y a partir de ese punto empieza a
aumentar más o menos linealmente. El resultado es que discernimos bien las
frecuencias medias pero tenemos más dificultad discerniendo las altas o bajas.

\paragraph{Problema 4.}

En condiciones escotópicas, las células receptoras que más se usan son los
\emph{bastones}. Son muchísimo más sensibles que los conos, y estan más
dispersos por la retina (aportando mayor campo visual), pero por contra no
permiten ningún tipo de percepción al color. Además, su baja densidad sumado
al hecho de que los impulsos de varios de ellos se agrupan en el nervio óptico
reduce considerablemente la nitidez.

\paragraph{Problema 5.}

La ley de Weber-Fechner indica que el incremento de percepción de brillo es
proporcional al ratio entre la poténcia anterior y la nueva. Dicho de otra
forma: nuestra noción de brillo es logarítmica. Ante una fuente puntual que sube
su potencia de forma lineal percibiremos un brillo que sube muy rápido al
principio y luego apenas cambia.

\finishpage


\startpage
\paragraph{Problema 6.}

\subparagraph{Apartado A.}

Cuantificación \emph{mid-tread} ya que tiene nivel cero. Hay 5 niveles por
codificar, con lo cual necesitamos \textsf{Nbits} = 3 bits como mínimo.

\subparagraph{Apartado B.}

En orden:

\begin{itemize}
\item Máximo absoluto del error de cuantificación.
\item Mínimo absoluto del error de cuantificación.
\item Integral cuadrática (energía) del error de cuantificación.
\item Integral cuadrática (energía) de la señal original.
\item SNR expresado en decibelios.
\end{itemize}

Los máximos y mínimos absolutos corresponden a medio paso, como se espera
en un cuantificador. El SNR bajo indica que la cuantificación es muy mala
(introduce mucho ruido), lo cual es consistente ya que el paso es muy grande.

\finishpage
\end{document}
