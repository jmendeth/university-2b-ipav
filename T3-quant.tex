\documentclass[catalan,border=15pt,class=scrartcl,multi=minipage,parskip=half*]{standalone}

% encoding
\usepackage[utf8]{inputenc}
\usepackage[T1]{fontenc}
\usepackage{lmodern}
\usepackage{babel}

% formatting and fixes
\frenchspacing
\usepackage[style=spanish]{csquotes}
\MakeAutoQuote{«}{»}
\usepackage{bookmark}

% ADD ANY SPECIFIC PACKAGES HERE
% (CHEMISTRY, CODE, PUBLISHING)
\usepackage[usenames,dvipsnames,svgnames,table]{xcolor}
\usepackage{adjustbox}
\usepackage{booktabs}
\usepackage{mathtools}
\usepackage{commath}
\usepackage{tikz}
\usepackage{siunitx}
\usepackage{nicefrac}
\usetikzlibrary{calc}
\usetikzlibrary{arrows.meta}
\usetikzlibrary{automata}
\usepackage{minted}

% other options
\setcounter{tocdepth}{6}
\setcounter{secnumdepth}{2}

% hyperlink setup / metadata
\usepackage{hyperref}
\AfterPreamble{\hypersetup{
  pdfauthor={Xavier Mendez},
  pdfsubject={IPAV},
  pdfpagelayout=OneColumn,
}}

% custom commands
\newcommand{\startpage}{\begin{minipage}{30em} \setlength{\parskip}{0.5em}}
\newcommand{\finishpage}{\end{minipage}}
\newcommand{\iopair}[2]{\( \left(#1\right) \rightarrow #2 \)}

\AfterPreamble{\hypersetup{
  pdftitle={Entregable 3: Cuantificación 1D e histograma de imágenes},
}}

\begin{document}
\startpage

\paragraph{Problema 1.}

\subparagraph{Apartado A.}

La potencia de la señal original corresponde al segundo momento de la
función de densidad:
%
\begin{align*}
  P &= \int^{\infty}_{-\infty} x^2 f(x) \dif x
\\
  P &=
    \int^{-\nicefrac{1}{4}}_{-1} x^2 \, \num{0.1} \dif x +
    \int^{\nicefrac{1}{4}}_{-\nicefrac{1}{4}} x^2 \, \num{1.97} \dif x +
    \int^{1}_{\nicefrac{1}{4}} x^2 \, \num{0.1} \dif x
\\+
  P &=
    \num{0.1} \left. \frac{x^3}{3} \right|^{-\nicefrac{1}{4}}_{-1} +
    \num{1.97} \left. \frac{x^3}{3} \right|^{\nicefrac{1}{4}}_{-\nicefrac{1}{4}} +
    \num{0.1} \left. \frac{x^3}{3} \right|^{1}_{\nicefrac{1}{4}}
\\
  P &=
    \num{0.1} \left. \frac{x^3}{3} \right|^{-\nicefrac{1}{4}}_{-1} +
    \num{1.97} \left. \frac{x^3}{3} \right|^{\nicefrac{1}{4}}_{-\nicefrac{1}{4}} +
    \num{0.1} \left. \frac{x^3}{3} \right|^{1}_{\nicefrac{1}{4}}
    = \frac{13}{480} \simeq \num{27.09e-3}
\end{align*}

\subparagraph{Apartado B.}

La aproximación habitual para la potencia del error en cuantificadores \emph{mid-rise}:
%
\begin{align*}
  \sigma_q^2 = \frac{X_{max}^2}{3 \cdot 2^{2B}}
           = \frac{1^2}{3 \cdot 2^{2 \cdot 5}}
           \simeq \num{0.326e-3}
\end{align*}
%
Nos devuelve el resultado exacto en este caso porque la función de densidad nos
permite dividir la distribución de la señal en ocho tramos uniformes, y el
número de niveles del cuantificador es múltiplo de ocho.

\subparagraph{Apartado C.}

Por inspección visual, el intervalo $\left|x\right| \leq \num{0.25}$ queda
convertido en $\left|y\right| \leq \num{0.75}$ al pasar por la función de
compresión, y se cuantifica con $\frac{\num{0.75}}{1} \, 2^5 = 24$ niveles.

Los otros 8 niveles se emplean para $\left|x\right| > \num{0.25}$.

\subparagraph{Apartado D.}

El paso en el intervalo $\left|x\right| \leq \num{0.25}$ es de
$\frac{2 \cdot \num{0.25}}{24} \approx \num{0.0208}$ y en el resto es de
$\frac{2 - 2 \cdot \num{0.25}}{8} = \num{0.1875}$.

\subparagraph{Apartado E.}

Una manera informal de hacerlo es calcular la potencia del error,
condicionado a $\left|x\right| \leq \num{0.25}$ (donde podemos considerar un
cuantificador uniforme y aplicar su expresión):
%
\begin{align*}
  \Bigl. \sigma_q^2 \Bigr|_{\left|x\right| \leq \num{0.25}}
    &= \frac{\num{.25}^2}{3 \cdot 24^2} = \frac{1}{27648}
\end{align*}
%
Hacemos lo mismo con el conjunto restante:
%
\begin{align*}
  \Bigl. \sigma_q^2 \Bigr|_{\left|x\right| > \num{0.25}}
    &= \frac{\num{.75}^2}{3 \cdot 8^2} = \frac{3}{1024}
\end{align*}
%
Teniendo en cuenta las probabilidades de entrar en cada caso,
la potencia media del error es:
%
\begin{align*}
  \sigma_q^2 = \num{0.015} \, \frac{3}{1024} + \num{0.985} \, \frac{1}{27648}
             \simeq \num{0.080e-3}
\end{align*}

\finishpage
\end{document}
