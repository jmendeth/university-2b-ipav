\documentclass[catalan,border=15pt,class=scrartcl,multi=minipage,parskip=half*]{standalone}

% encoding
\usepackage[utf8]{inputenc}
\usepackage[T1]{fontenc}
\usepackage{lmodern}
\usepackage{babel}

% formatting and fixes
\frenchspacing
\usepackage[style=spanish]{csquotes}
\MakeAutoQuote{«}{»}
\usepackage{bookmark}

% ADD ANY SPECIFIC PACKAGES HERE
% (CHEMISTRY, CODE, PUBLISHING)
\usepackage[usenames,dvipsnames,svgnames,table]{xcolor}
\usepackage{adjustbox}
\usepackage{booktabs}
\usepackage{mathtools}
\usepackage{commath}
\usepackage{tikz}
\usepackage{siunitx}
\usepackage{nicefrac}
\usetikzlibrary{calc}
\usetikzlibrary{arrows.meta}
\usetikzlibrary{automata}
\usepackage{minted}

% other options
\setcounter{tocdepth}{6}
\setcounter{secnumdepth}{2}

% hyperlink setup / metadata
\usepackage{hyperref}
\AfterPreamble{\hypersetup{
  pdfauthor={Xavier Mendez},
  pdfsubject={IPAV},
  pdfpagelayout=OneColumn,
}}

% custom commands
\newcommand{\startpage}{\begin{minipage}{30em} \setlength{\parskip}{0.5em}}
\newcommand{\finishpage}{\end{minipage}}
\newcommand{\iopair}[2]{\( \left(#1\right) \rightarrow #2 \)}

\AfterPreamble{\hypersetup{
  pdftitle={Entregable 3: Cuantificación 1D e histograma de imágenes},
}}

\begin{document}
\startpage

\paragraph{Problema 1.}

\subparagraph{Apartado A.}

La potencia de la señal original corresponde al segundo momento de la
función de densidad:
%
\begin{align*}
  P &= \int^{\infty}_{-\infty} x^2 f(x) \dif x
\\
  P &=
    \int^{-\nicefrac{1}{4}}_{-1} x^2 \, \num{0.1} \dif x +
    \int^{\nicefrac{1}{4}}_{-\nicefrac{1}{4}} x^2 \, \num{1.97} \dif x +
    \int^{1}_{\nicefrac{1}{4}} x^2 \, \num{0.1} \dif x
\\
  P &=
    \num{0.1} \left. \frac{x^3}{3} \right|^{-\nicefrac{1}{4}}_{-1} +
    \num{1.97} \left. \frac{x^3}{3} \right|^{\nicefrac{1}{4}}_{-\nicefrac{1}{4}} +
    \num{0.1} \left. \frac{x^3}{3} \right|^{1}_{\nicefrac{1}{4}}
\\
  P &=
    \num{0.1} \left. \frac{x^3}{3} \right|^{-\nicefrac{1}{4}}_{-1} +
    \num{1.97} \left. \frac{x^3}{3} \right|^{\nicefrac{1}{4}}_{-\nicefrac{1}{4}} +
    \num{0.1} \left. \frac{x^3}{3} \right|^{1}_{\nicefrac{1}{4}}
    = \frac{13}{480} \simeq \num{27.09e-3}
\end{align*}

\subparagraph{Apartado B.}

La aproximación habitual para la potencia del error en cuantificadores \emph{mid-rise}:
%
\begin{align*}
  \sigma_q^2 = \frac{X_{max}^2}{3 \cdot 2^{2B}}
           = \frac{1^2}{3 \cdot 2^{2 \cdot 5}}
           \simeq \num{0.326e-3}
\end{align*}
%
Nos devuelve el resultado exacto en este caso porque la función de densidad nos
permite dividir la distribución de la señal en ocho tramos uniformes, y el
número de niveles del cuantificador es múltiplo de ocho.

\subparagraph{Apartado C.}

Por inspección visual, el intervalo $\left|x\right| \leq \num{0.25}$ queda
convertido en $\left|y\right| \leq \num{0.75}$ al pasar por la función de
compresión, y se cuantifica con $\frac{\num{0.75}}{1} \, 2^5 = 24$ niveles.

Los otros 8 niveles se emplean para $\left|x\right| > \num{0.25}$.

\subparagraph{Apartado D.}

El paso en el intervalo $\left|x\right| \leq \num{0.25}$ es de
$\frac{2 \cdot \num{0.25}}{24} \approx \num{0.0208}$ y en el resto es de
$\frac{2 - 2 \cdot \num{0.25}}{8} = \num{0.1875}$.

\subparagraph{Apartado E.}

Una manera informal de hacerlo es calcular la potencia del error,
condicionado a $\left|x\right| \leq \num{0.25}$ (donde podemos considerar un
cuantificador uniforme y aplicar su expresión):
%
\begin{align*}
  \Bigl. \sigma_q^2 \Bigr|_{\left|x\right| \leq \num{0.25}}
    &= \frac{\num{.25}^2}{3 \cdot 24^2} = \frac{1}{27648}
\end{align*}
%
Hacemos lo mismo con el conjunto restante:
%
\begin{align*}
  \Bigl. \sigma_q^2 \Bigr|_{\left|x\right| > \num{0.25}}
    &= \frac{\num{.75}^2}{3 \cdot 8^2} = \frac{3}{1024}
\end{align*}
%
Teniendo en cuenta las probabilidades de entrar en cada caso,
la potencia media del error es:
%
\begin{align*}
  \sigma_q^2 = \num{0.015} \, \frac{3}{1024} + \num{0.985} \, \frac{1}{27648}
             \simeq \num{0.080e-3}
\end{align*}

\finishpage
\startpage

\paragraph{Problema 2.}

\subparagraph{Apartado A.}

Se desea una función de $\left[0, 1\right]$ a $\left[0, 1\right]$ con
pendiente alta ($\left|T_1'(r)\right| > 1$) para $r$ cercanos a $0$, y pendiente
baja ($\left|T_1'(r)\right| < 1$) para $r$ cercanos a $1$. Una posible solución
es:
%
\begin{align*}
  T_1(r) = \sqrt{r}
\end{align*}

\subparagraph{Apartado B.}

Se define una función con dos tramos. En el primero, cuando $r$ está en $\left[
\num{0}, \num{0.6}\right]$, se devuelve siempre $0$. En el segundo, se devuelve
$r$ ya que no hay que modificar nada:
%
\begin{align*}
  T_2(r) = \begin{cases}
    0 & \text{si $r \in \left[\num{0}, \num{0.6}\right]$} \\
    r & \text{otro caso}
  \end{cases}
\end{align*}

\subparagraph{Apartado C.}

Lo más sencillo es construir una función que comprima todos los valores,
uniformemente ($T_3'(r) = k, \, k < 1$). Como no lo especifican, haremos la
compresión hacia el centro ($T_3(\nicefrac{1}{2}) = \nicefrac{1}{2}$).
%
\begin{align*}
  T_3(r) = k\left(r - \nicefrac{1}{2}\right) + \nicefrac{1}{2}
\end{align*}
%
con $k \in \left(0, 1\right)$ dependiendo del contraste deseado.

\finishpage
\startpage

\paragraph{Problema 3.}

Los niveles (no vacíos) que caen dentro del primer tramo serían el 0 (12),
4 (4), 8 (12), 12 (4), 16 (12) y se suman todos en el nivel~0 de la imagen
resultante: $h_y[0] = 44$.

Los niveles que caen en el segundo tramo serían el
20 (4), 24 (12), 28 (4) y se suman en el nivel 31 de la imagen resultante:
$h_y[31] = 20$.

El histograma queda así:
%
\begin{align*}
  h_y[r] = \begin{cases}
    44 & \text{si $r = 0$} \\
    20 & \text{si $r = 31$} \\
    0 & \text{otro caso}
  \end{cases}
\end{align*}

\finishpage
\startpage

\paragraph{Problema 4.}

Empezaremos calculando la función de probabilidad:
%
\begin{align*}
  F_x(r) &= \int_{-\infty}^r f_x(r) \dif r
\\
  F_x(r) &= \begin{dcases}
    0 & \text{si $r < 0$}
  \\
    \int_0^r 4r \dif r
        & \text{si $r \in \left[0, \nicefrac{1}{2}\right)$}
  \\
    \int_0^{\nicefrac{1}{2}} 4r \dif r + \int_{\nicefrac{1}{2}}^{r} -4\left(r-1\right) \dif r
        & \text{si $r \in \left[\nicefrac{1}{2}, 1\right)$}
  \\
    1 & \text{si $r \geq 1$}
  \end{dcases}
\\
  F_x(r) &= \begin{dcases}
    0 & \text{si $r < 0$}
  \\
    \bigl. 2 r^2 \bigr|_0^r
        & \text{si $r \in \left[0, \nicefrac{1}{2}\right)$}
  \\
    \bigl. 2 r^2 \bigr|_0^{\nicefrac{1}{2}} \; + \; \bigl. -2r^2 +4r\bigr|_{\nicefrac{1}{2}}^{r}
        & \text{si $r \in \left[\nicefrac{1}{2}, 1\right)$}
  \\
    1 & \text{si $r \geq 1$}
  \end{dcases}
\\
  F_x(r) &= \begin{dcases}
    0 & \text{si $r < 0$}
  \\
    2 r^2
        & \text{si $r \in \left[0, \nicefrac{1}{2}\right)$}
  \\
    1 - 2 \left(r-1\right)^2
        & \text{si $r \in \left[\nicefrac{1}{2}, 1\right)$}
  \\
    1 & \text{si $r \geq 1$}
  \end{dcases}
\end{align*}

\subparagraph{Apartado A.}

La transformación que ecualiza la imagen es precisamente $F_x(r)$, ya que
al aplicarla la distribución resultante es
$F_y(r) = F_x\left(F_x^{-1}(r)\right) = r$ (distribución uniforme).

\subparagraph{Apartado B.}

La pendiente de la transformación (que es $f_x(r)$) nos indica que los niveles
$\left[0, \num{0.25}\right]$, $\left[\num{0.75}, 1\right]$ (donde la pendiente
es mayor a la unidad) se comprimirán, especialmente los extremos. Así mismo,
los niveles $\left[\num{0.25}, \num{0.75}\right]$ se expandirán, especialmente
la parte central.

\subparagraph{Apartado C.}

Ya que la imagen está ecualizada, la distribución es ahora uniforme y el
histograma debería ser constante para cualquier $r$: $H_x[r] = k$.


\finishpage
\end{document}
